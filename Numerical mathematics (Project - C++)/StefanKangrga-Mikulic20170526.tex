\documentclass[12pt,leqno,a4paper]{article}

\usepackage{epsf}
\usepackage{amsmath}
\usepackage{amsfonts}
\usepackage[T1]{fontenc}
\usepackage[T2A]{fontenc}
\usepackage[utf8]{inputenc}
\usepackage{multirow}
\usepackage{hyperref}
\usepackage[english,serbianc]{babel}
\usepackage{listings} %!!!!!!!!!!!!!!!!!!!!!!
\usepackage{pgfplots}
 
\pgfplotsset{compat = newest}

\usepackage{pgf,tikz}
\usepackage{mathrsfs}
\usetikzlibrary{arrows}
\pagestyle{empty}


\def\dj{d{\hspace{-.32em}\raisebox{.7ex}{-}}}
\def\Dj{D{\hspace{-.75em}\raisebox{.3ex}{-}\hspace{.4em}}}
\def\boxx{\mbox{\footnotesize \fbox{${\;\;}^{\!}$}}}
\def\stop{\mbox{\footnotesize {\vrule width 6pt height 6pt}}}

\newtheorem{zadatak1}{Zadatak 1.5}
\newtheorem{zadatak2}{Zadatak 2.5}

% Custom colors
\usepackage{color}
\definecolor{deepblue}{rgb}{0,0,0.5}
\definecolor{deepred}{rgb}{0.6,0,0}
\definecolor{deepgreen}{rgb}{0,0.5,0}
\definecolor{codegreen}{rgb}{0,0.6,0}%%%%%%%%%%%%%%%%%%%%%%%%%%%%%%%%%
\definecolor{codegray}{rgb}{0.5,0.5,0.5}%%%%%%%%%%%%%%%%%%%%%%%%%%%%%
\definecolor{codepurple}{rgb}{0.58,0,0.82}%%%%%%%%%%%%%%%%%%%%%%%%%%%
\definecolor{backcolour}{rgb}{0.95,0.95,0.92}%%%%%%%%%%%%%%%%%%%%%%%%%

\renewcommand{\refname}{Литература}

\renewcommand{\figurename}{Слика}

\usepackage{listings}

% Python style for highlighting
\newcommand\pythonstyle{\lstset{
		language=Python,
		basicstyle=\ttm,
		otherkeywords={self},             % Add keywords here
		keywordstyle=\ttb\color{deepblue},
		emph={MyClass,__init__},          % Custom highlighting
		emphstyle=\ttb\color{deepred},    % Custom highlighting style
		stringstyle=\color{deepgreen},
		frame=tb,                         % Any extra options here
		showstringspaces=false            % 
}}


% Python environment
\lstnewenvironment{python}[1][]
{
	\pythonstyle
	\lstset{#1}
}
{}

% Python for external files
\newcommand\pythonexternal[2][]{{
		\pythonstyle
		\lstinputlisting[#1]{#2}}}

\title{\color{deepblue}  NAD-Numeri\v cka analiza\\* Predispitni rad }
\author{\color{deepred} Stefan Kangrga-Mikuli\' c, 2017/526 \\* \color{deepred} Fizi\v cka elektronika 13E082NUM, Elektrotehni\v cki fakultet, Beograd }
\date{16.09.2021.godine}
%%%%%%%%%%%%%%%%%%%%%%%%%%%%%%%%%%%%%%%%%%%%%%%%%%%%%%%%%%%%%%%%%%%%%%%%%%%%%%%%%%%%%%%%%
%Code listing style named "mystyle"
\lstdefinestyle{mystyle}{
  backgroundcolor=\color{backcolour}, commentstyle=\color{codegreen},
  keywordstyle=\color{magenta},
  numberstyle=\tiny\color{codegray},
  stringstyle=\color{codepurple},
  basicstyle=\ttfamily\footnotesize,
  breakatwhitespace=false,         
  breaklines=true,                 
  captionpos=b,                    
  keepspaces=true,                 
  numbers=left,                    
  numbersep=5pt,                  
  showspaces=false,                
  showstringspaces=false,
  showtabs=false,                  
  tabsize=2
}

%"mystyle" code listing set
\lstset{style=mystyle}
%%%%%%%%%%%%%%%%%%%%%%%%%%%%%%%%%%%%%%%%%%%%%%%%%%%%%%%%%%%%%%
\begin{document}
\maketitle

\v Clanovi tima: Stefan Kangrga-Mikuli\' c, 2017/526

\section*{Zadatak 1.5} 

Izlazna struja solarne \' celije zavisi od napona. Napon $V_{mp}$ pri kome je izlazna struja maksimalna je dat jedna\v cinom:

\begin{equation*}
  \begin{aligned}
   e^{\frac{q \cdot V_{mp}}{k_{B} \cdot T}} \cdot (1 + \frac{q \cdot V_{mp}}{k_{B} \cdot T}) = \frac{q \cdot V_{OC}}{k_{B} \cdot T}
 \end{aligned}
\end{equation*}

gde je $V_{OC}$ napon otvorenog kola, $T$ temperatura u kelvinima, $q = 1,6022 \cdot 10^{-19} C$ naelektrisanje elektrona, $k_{B} = 1.3806 \cdot 10^{-23} J/k$ Bolcmanova konstanta. Neka je $V_{OC} = 0.5 V$ i $T=297 K$. Razviti program koji izra\v cunava napon $V_{mp}$ za koji je izlazna struja iz solarne \' celije maksimalna. Primeniti metodu proste iteracije sa po\v cetnom iteracijom $V_{mp}^{0}=0.5 V$ sa ta\v cno\v s\' cu $10^{-3}$.

\section{Opis algoritma}%%%%%%%%%%%%%%%%%%%%%%%%%%%%%%%%%%%%%%%%%%

Na po\v cetku mora da se preformuli\v se polazna jedna\v cina na oblik:

\begin{equation*}
  \begin{aligned}
    g(x)=x
 \end{aligned}
\end{equation*}

za neku funkciju $g$, ali tako da za svako $x \in [a,b]$, gde je interval $[a, b]$ gde se nalazi pribli\v zna vrednost, va\v zi da je $a \leq g(x) \leq b$ i da je $\mid g^{\prime}(x) \mid \leq k$ pri \v cemu je $0<k<1$.
Od polazne jedna\v cine se dobija:

\begin{equation*}
  \begin{aligned}
   V_{mp}=\frac{k_{B} \cdot T}{2 \cdot q} \cdot (\frac{q \cdot V_{OC}}{k_{B} \cdot T} - \ln{\frac{q \cdot V_{mp}}{k_{B} \cdot T}})
 \end{aligned}
\end{equation*}
tj. 
\begin{equation*}
  \begin{aligned}
   g(x)=\frac{k_{B} \cdot T}{2 \cdot q} \cdot (\frac{q \cdot V_{OC}}{k_{B} \cdot T} - \ln{\frac{q \cdot x}{k_{B} \cdot T}})
 \end{aligned}
\end{equation*}
 


Iterativna metoda glasi:
\begin{equation*}
  \begin{aligned}
   x_{n+1}=\frac{k_{B} \cdot T}{2 \cdot q} \cdot (\frac{q \cdot V_{OC}}{k_{B} \cdot T} - \ln{\frac{q \cdot x_{n}}{k_{B} \cdot T}})
 \end{aligned}
\end{equation*}

Deo koda koji vr\v si iteraciju :
\begin{lstlisting}[language=C++ ]
while (radi) {
		prethodni_rezultat = pomocna_promenljiva;
		pomocna_promenljiva = (kB * T) / (2 * q) *((q * Voc) / (kB * T) - log((q * pomocna_promenljiva) / (kB * T)));
		broj_iteracije++;
		cout << broj_iteracije << "		" << pomocna_promenljiva << endl;
		if (broj_iteracije == 10 || radi == uporedi(pomocna_promenljiva, prethodni_rezultat)) { break; }
	}
\end{lstlisting}

Za $x \in [0,0.5]$ va\v zi da je $\mid g^{\prime}(x) \mid = \mid \frac{k_{B} \cdot T}{2 \cdot q \cdot x}\mid \leq 0.0256 < 1$. Sada su ispunjeni svi uslovi da se primeni ova iterativna metoda sa $x_{0}=V_{mp}^{0}=0.5$.

Za kriterijum zaustavljanja nalazimo:
\begin{equation*}
  \begin{aligned}
  \frac{0.1^{n}}{1-0.1} \cdot \mid 0.211966 - 0.5 \mid \leq 10^{-3}
 \end{aligned}
\end{equation*}
da je za $n=3$ ispunjena ocena.

\begin{table}[h!]
\centering
 \begin{tabular}{||c c||} 
 \hline
 n & x_{n}(=g(x_{n-1})) \\ [0.5ex] 
 \hline\hline
 0 & 0.5 \\ 
 1 & 0.211966\\ 
 2 & 0.222947\\ 
 3 & 0.222301\\ 
 4 & 0.222338\\ 
 5 & 0.222336\\ 
 6 & 0.222336\\ 
 \hline
 \end{tabular}
\end{table}


\section{Komentari}

Imaju\' ci u vidu na zahtevanu ta\v cnost, poklapanja vrednosti na 4 decimale je do\v slo ve\' c u 3. i 4.iteraciji,pa je program tako napisan da ipak radi dalje dok ne dodje do poklapanja  na 6 decimale.

\section{Dodatak}
Main.cpp
\begin{lstlisting}[language=C++ ]
#include <cmath>
#include <iostream>
using namespace std;

bool uporedi(float trenutni_rezultat, float prethodni_rezultat) {
	if (fabs(trenutni_rezultat - prethodni_rezultat) < 0.000001f) {
		return true;
	}
	return false;
}

int main() {
	float q = 0.00000000000000000016022;
	float kB = 0.000000000000000000000013806;
	float Voc = 0.5;
	float T = 297;
	float Vmp_pocetno = 0.5;
	float pomocna_promenljiva = 0, prethodni_rezultat = 0;
	int broj_iteracije = 0;
	bool radi = true;

	cout << "Pocetna jednacina se moze tranformisati u: Vmp=\n " <<  (kB * T)/(2*q) << " * ( " << (q * Voc) / (kB * T) << " - ln( " << q / (kB * T) << "*Vmp))" << endl << endl;
	pomocna_promenljiva = Vmp_pocetno;

	cout << broj_iteracije << "		" << pomocna_promenljiva << endl;
	while (radi) {
		prethodni_rezultat = pomocna_promenljiva;
		pomocna_promenljiva = (kB * T) / (2 * q) *((q * Voc) / (kB * T) - log((q * pomocna_promenljiva) / (kB * T)));
		broj_iteracije++;
		cout << broj_iteracije << "		" << pomocna_promenljiva << endl;
		if (broj_iteracije == 10 || radi == uporedi(pomocna_promenljiva, prethodni_rezultat)) { break; }
		//radi = uporedi(pomocna_promenljiva, prethodni_rezultat);
	}

}
\end{lstlisting}
\section*{Zadatak 2.5} 

Struje $i_1,i_2,i_3,i_4,i_5 $ u kolu prikazanom na slici 3, mogu se odrediti re\v savanjem sistema jednacina:

\begin{equation*}
  \begin{aligned}
    9.5i1-2.5i2-2i4=12 \\
    -2.5i1+11i2-3.5i3-5i5=-16\\
    -3.5i2+15.5i3-4i5=14\\
    -2i1+7i4-3i5=10\\
    -5i2-4i3-3i4+12i5=-30
 \end{aligned}
\end{equation*}

Razviti program koji izra\v cunava struje primenom Gausove metode eliminacije sa izborom pivota i $LU$ dekompozicije. Razviti program koji primenjuje Gaus-Zajdelovu metodu i ispitati da li ova metoda konvergira za po\v cetnu iteraciju koja je jednaka nula-vektoru. Komentarisati dobijene rezultate.

\section{Opis algoritma}%%%%%%%%%%%%%%%%%%%%%%%%%%%%%%%%%%%%%%%%%%%%%%%%%%%%%%%%%%%%%%%%%%%%%%%%%%%%%%%

Program, koji simulira re\v savanje zadatka pomo\'cu Gausove metode eliminacije i
Gaus-Zajdelove metode, radjen je na programskom jeziku C++.
Program se sastoji iz nekoliko fajlova( Jednacina.h, Jednacina.cpp, SistemJednacina.h, SistemJednacina.cpp i Main.cpp).
Klasa "Jedna\v cina" \  ima promenljivu "koeficijenti\_"\  \v sto predstavlja vektor
elemenata \v ciji \v clanovi ce biti koeficijente date jedna\v cine.
Klasa "SistemJednacina"\  ima promenljivu "jednacine\_" ( predstavlja vektor
\v ciji \v clanovi ce biti objekti klase "Jedna\v cina"\  i predstavlja\' ce odredjenu vrstu u matrici sistema jedna\v cine),
"multiplikatori\_"\\( predstavlja vektor tipa "double"\  \v sto \' ce predstavljati multiplikatore i ovaj vektor \' ce se
prazniti pri svakom novom koraku pri odredjivanju nove matrice), "resenje\_Gaus\_",  "resenje\_Zajdel\_"( vektore gde \v cuvamo rezultate programa).\\* 

\subsection{Gausova metoda eliminacije}
Formiramo pro\v sirenu matricu sistema jednacine:\\*

\begin{bmatrix}
 A,b  
\end{bmatrix}
$=$
\left\lceil
\begin{matrix}
 9.5 & -2.5 & 0 & -2 & 0\\
 -2.5 & 11 & -3.5 & 0 & -5\\
 0 & -3.5 & 15.5 & 0 & -4\\
 -2 & 0 & 0 & 7 & -3\\
 0 & -5 & -4 & -3 & 12
\end{matrix}
\left\lvert
\begin{matrix}
 12\\
 -16\\
 14\\
 10\\
 -30
\end{matrix}
\right\rceil 

Za prvu kolonu biramo pivot tako \v sto nalazimo po apsolutnoj vrednosti najve\' ci element u prvoj koloni. Kako je najve\' ci pivot u 1.vrsti, ne menjamo mesta vrstama.
Deo programa koji radi pivotiranje vrste jedna\v cina se zasniva na tome sto zamenjuje mesta promenljive tipa "Jedna\v cina" u vektoru "jedna\v cine_" i tako simulira menjanje vrsta u sistemu jedna\v cina:\\


\begin{lstlisting}[language=C++ ]
void SistemJednacina::pivotiranje(int broj_kolone){
  int maks = broj_kolone;
  for (int i = maks; i+1< 5; i++) {
	if (abs(jednacine_[maks]->koeficijenti_[broj_kolone]) < abs(jednacine_[i + 1]->koeficijenti_[broj_kolone])) {
		maks = i + 1;
	}
  }
  if (maks != broj_kolone) {
	Jednacina* jednacina = new Jednacina();
	jednacina->koeficijenti_.push_back(jednacine_[maks]->koeficijenti_[0]);
	jednacina->koeficijenti_.push_back(jednacine_[maks]->koeficijenti_[1]);
	jednacina->koeficijenti_.push_back(jednacine_[maks]->koeficijenti_[2]);
	jednacina->koeficijenti_.push_back(jednacine_[maks]->koeficijenti_[3]);
	jednacina->koeficijenti_.push_back(jednacine_[maks]->koeficijenti_[4]);
	jednacina->koeficijenti_.push_back(jednacine_[maks]->koeficijenti_[5]);

	jednacine_[maks]->koeficijenti_[0] = jednacine_[broj_kolone]->koeficijenti_[0];
	jednacine_[maks]->koeficijenti_[1] = jednacine_[broj_kolone]->koeficijenti_[1];
	jednacine_[maks]->koeficijenti_[2] = jednacine_[broj_kolone]->koeficijenti_[2];
	jednacine_[maks]->koeficijenti_[3] = jednacine_[broj_kolone]->koeficijenti_[3];
	jednacine_[maks]->koeficijenti_[4] = jednacine_[broj_kolone]->koeficijenti_[4];
	jednacine_[maks]->koeficijenti_[5] = jednacine_[broj_kolone]->koeficijenti_[5];

	jednacine_[broj_kolone]->koeficijenti_[0] = jednacina->koeficijenti_[0];
	jednacine_[broj_kolone]->koeficijenti_[0] = jednacina->koeficijenti_[1];
	jednacine_[broj_kolone]->koeficijenti_[0] = jednacina->koeficijenti_[2];
	jednacine_[broj_kolone]->koeficijenti_[0] = jednacina->koeficijenti_[3];
	jednacine_[broj_kolone]->koeficijenti_[0] = jednacina->koeficijenti_[4];
	jednacine_[broj_kolone]->koeficijenti_[0] = jednacina->koeficijenti_[5];

  }
}
\end{lstlisting}
%\begin{lstlisting}[language=C++ ]
%\end{lstlisting}
Slede\' ce se nalaze modifikatori za $k$-tu vrstu ($k$=2,3,4,5):
\begin{equation*}
  \begin{aligned}
    m_{21}=a_{21}/a_{11}=-2.5/9.5=-0.263158\\
    m_{31}=a_{31}/a_{11}=0/9.5=0\\
    m_{41}=a_{41}/a_{11}=-2/9.5=-0.210526\\
    m_{51}=a_{51}/a_{11}=0/9.5=0\\
\end{aligned}
\end{equation*}

Deo programa koji nalazi nove multiplikatore tako \v sto u vektor tipa $double$ dodaju elementi koje predstavljaju nove multiplikatore koji se dobijaju deljenjem odgovaraju\' cih \v clanova vektora koeficijenata(predstavlja kolonu) sa odgovaraju\' cim \v clanovima vektora jedna\cv cina(predstavlja vrstu):

\begin{lstlisting}[language=C++ ]
void SistemJednacina::noviMultiplikatori(int broj_vrste){
	int pocetni_red = broj_vrste - 1;
	for (int i = broj_vrste; i < 5; i++) {
		multiplikatori_.push_back(jednacine_[i]->koeficijenti_[broj_vrste-1]/ jednacine_[pocetni_red]->koeficijenti_[broj_vrste-1]);
	}
}
\end{lstlisting}

Sada odgovaraju\' ci elementi odgovaraju\' cih vrsta se oduzimaju sa odgovaraju\' cim elementima prve vrste koja je prethodno pomno\v zena odgovaraju\' cim multiplikatorom:

\begin{equation*}
  \begin{aligned}
    a_{ij}^{2}=a_{ij}-m_{i1} \cdot a_{1j} \\
    b_{i}^{2}=b_{i}-m_{i1} \cdot b_{1}
 \end{aligned}
\end{equation*}

Deo programa koji sredjuje matricu tako \v sto menja \v clanove vektora promenljive "koeficijenti\_"\  za date clanove vektora promenljive "jednacine\_":

\begin{lstlisting}[language=C++ ]
void SistemJednacina::novaMatrica(int broj_vrste){
	int pocetni_red = broj_vrste-1, k=0;
	for (int i = broj_vrste; i < 5; i++) {
		for (int j = broj_vrste-1; j < 6; j++) {
			jednacine_[i]->koeficijenti_[j] = jednacine_[i]->koeficijenti_[j] - multiplikatori_[k] * jednacine_[pocetni_red]->koeficijenti_[j];
		}
		k++;
	}
}
\end{lstlisting}


Ponavljamo iste postupke( biranje pivota, zamena mesta vrste matrica, nala\v zenje multiplikatora i oduzimanje vrsta) na sistemu od 4 jedna\v cine \v sto je za jedan manje nego prethodni korak, i tako sve dok se ne formira gornja
trougaona matrica \v cije se kona\v cno re\v senje dobija postupkom re\v savanjem unazad:

\begin{equation*}
  \begin{aligned}
\begin{equation*}
  \begin{aligned}
    x_{n}=\frac{b_{n}}{a_{nn}}
    \end{aligned}
\end{equation*}

\begin{equation*}
  \begin{aligned}
    x_{i}=\frac{1}{a_{ii}}(b_{i}-\sum_{\substack{i+1<k<n
  }} a_{ik} \cdot x_{k})
  \end{aligned}
\end{equation*}

Deo programa koji nalaze od poslednje date matrice kona\v cna re\v senja:

\begin{lstlisting}[language=C++ ]
void SistemJednacina::nalazenjeResenja(){
	double sum = 0;
	
	for (int i = 4; i >= 0; i--) {
		sum = 0;
		if (i == 4) {
			resenje_Gaus_.push_back(jednacine_[i]->koeficijenti_[i+1] / jednacine_[i]->koeficijenti_[i]);
		}
		else {
			for (int j = i+1, k=1; j<5 ; j++,k++ ) {
				sum += (jednacine_[i]->koeficijenti_[j]* resenje_Gaus_[(int)(resenje_Gaus_.size())-k]);
			}
			resenje_Gaus_.push_back(1/(jednacine_[i]->koeficijenti_[i])*(jednacine_[i]->koeficijenti_[5] -sum));
		}
	}
}
}
\end{lstlisting}

%%%%%%%%%%%%%%%%%%%%%%%%%%%%%%%%%%%%%%%%%%%%%%%%%%%%%%%%%%%%%%%%%%%%%%%%%%%%%%%
\begin{bmatrix}
 A^{2},b^{2}  
\end{bmatrix}
$=$
\left\lceil
\begin{matrix}
 9.5 & -2.5 & 0 & -2 & 0\\
 0 & 10.3421 & -3.5 & -0.526316 & -5\\
 0 & -3.5 & 15.5 & 0 & -4\\
 0 & -0.526316 & 0 & 6.57895 & -3\\
 0 & -5 & -4 & -3 & 12
\end{matrix}
\left\lvert
\begin{matrix}
 12\\
 -12.8421\\
 14\\
 12.5263\\
 -30
\end{matrix}
\right\rceil 


\begin{equation*}
  \begin{aligned}
    m_{32}=a_{32}/a_{22}=-0.338422
    \end{aligned}
\end{equation*}
\begin{equation*}
  \begin{aligned}
    m_{42}=a_{42}/a_{22}=-0.0508906
    \end{aligned}
\end{equation*}
\begin{equation*}
  \begin{aligned}
    m_{52}=a_{52}/a_{22}=-0.483461\\
\end{aligned}
\end{equation*}

%%%%%%%%%%%%%%%%%%%%%%%%%%%%%%%%%%%%%%%%%%%%%%%%%%%%%%%%%%%%%%%%%%%%%%%%%%%%%%%
\begin{bmatrix}
 A^{3},b^{3}  
\end{bmatrix}
$=$
\left\lceil
\begin{matrix}
 9.5 & -2.5 & 0 & -2 & 0\\
 0 & 10.3421 & -3.5 & -0.526316 & -5\\
 0 & 0 & 14.3155 & -0.178117 & -5.69211\\
 0 & 0 & -0.178117 & 6.55216 & -3.25445\\
 0 & 0 & -5.69211 & -3.25445 & 9.5827
\end{matrix}
\left\lvert
\begin{matrix}
 12\\
 -12.8421\\
 9.65394\\
 11.8728\\
 -36.2087
\end{matrix}
\right\rceil 

\begin{equation*}
  \begin{aligned}
    m_{43}=a_{43}/a_{33}=-0.0124422
    \end{aligned}
\end{equation*}
    \begin{equation*}
  \begin{aligned}
    m_{53}=a_{53}/a_{33}=-0.397618
\end{aligned}
\end{equation*}

%%%%%%%%%%%%%%%%%%%%%%%%%%%%%%%%%%%%%%%%%%%%%%%%%%%%%%%%%%%%%%%%%%%%%%%%%%%%%%%
\begin{bmatrix}
 A^{4},b^{4}
\end{bmatrix}
$=$
\left\lceil
\begin{matrix}
 9.5 & -2.5 & 0 & -2 & 0\\
 0 & 10.3421 & -3.5 & -0.526316 & -5\\
 0 & 0 & 14.3155 & -0.178117 & -5.69211\\
 0 & 0 & 0 & 6.54995 & -3.32528\\
 0 & 0 & 0 & -3.32528 & 7.31941
\end{matrix}
\left\lvert
\begin{matrix}
 12\\
-12.8421\\
 9.65394\\
 11.9929\\
 -32.3701
\end{matrix}
\right\rceil 

\begin{equation*}
  \begin{aligned}
    m_{54}=a_{54}/a_{44}=-0.50768
\end{aligned}
\end{equation*}

%%%%%%%%%%%%%%%%%%%%%%%%%%%%%%%%%%%%%%%%%%%%%%%%%%%%%%%%%%%%%%%%%%%%%%%%%%%%%%%
\begin{bmatrix}
 A^{5},b^{5}  
\end{bmatrix}
$=$
\left\lceil
\begin{matrix}
 9.5 & -2.5 & 0 & -2 & 0\\
 0 & 10.3421 & -3.5 & -0.526316 & -5\\
 0 & 0 & 14.3155 & -0.178117 & -5.69211\\
 0 & 0 & 0 & 6.54995 & -3.32528\\
 0 & 0 & 0 & 0 & 5.63123
\end{matrix}
\left\lvert
\begin{matrix}
 12\\
-12.8421\\
 9.65394\\
 11.9929\\
 -26.2815
\end{matrix}
\right\rceil 

\begin{equation*}
  \begin{aligned}
    x_{5}=b_{5}/a_{55}=-2.5/9.5=-4.6671
    \end{aligned}
\end{equation*}
\begin{equation*}
  \begin{aligned}    
    x_{4}=\frac{1}{a_{44}}(b_{4}-a_{45} \cdot x_{5})=-0.5384
    \end{aligned}
\end{equation*}
\begin{equation*}
  \begin{aligned}
    x_{3}=\frac{1}{a_{33}}(b_{3}-a_{34} \cdot x_{4} - a_{35} \cdot x_{5})=-1.18805
    \end{aligned}
\end{equation*}
\begin{equation*}
  \begin{aligned}
    x_{2}=\frac{1}{a_{22}}(b_{2}- a_{23} \cdot x_{3} - a_{24} \cdot x_{4} - a_{25} \cdot x_{5})=-3.92755
    \end{aligned}
\end{equation*}
\begin{equation*}
  \begin{aligned}
    x_{1}=\frac{1}{a_{11}}(b_{1}- a_{12} \cdot x_{2} - a_{13} \cdot x_{3} - a_{14} \cdot x_{4} - a_{15} \cdot x_{5})=0.116244
    \end{aligned}
\end{equation*}
\end{aligned}
\end{equation*}
%%%%%%%%%%%%%%%%%%%%%%%%%%%%%%%%%%%%%%%%%%%%%%%%%%%%%%%%%%%%%%%%%%%%%%%%%%%%%%%
\subsection{Gaus-Zajdelova metoda}
Od polaznih jedna\v cina:
\begin{equation*}
  \begin{aligned}
    9.5i1-2.5i2-2i4=12 \\
    -2.5i1+11i2-3.5i3-5i5=-16\\
    -3.5i2+15.5i3-4i5=14\\
    -2i1+7i4-3i5=10\\
    -5i2-4i3-3i4+12i5=-30,
 \end{aligned}
\end{equation*}

potrebno je iz i-te jedna\v cine izraziti promenljivu $x_i$ i formiramo Gaus-Zajdelov iterativni proces:
\begin{equation*}
  \begin{aligned}
   x_{1}^{(k+1)}= \frac{2.5}{9.5} \cdot x_{2}^{(k)} + \frac{2}{9.5} \cdot x_{4}^{(k)}+ \frac{12}{9.5} \\
    \end{aligned}
\end{equation*}
\begin{equation*}
  \begin{aligned}
   x_{2}^{(k+1)}= \frac{2.5}{11} \cdot x_{1}^{(k+1)} + \frac{3.5}{11} \cdot x_{3}^{(k)} + \frac{5}{11} \cdot x_{5}^{(k)} - \frac{16}{11} \\
    \end{aligned}
\end{equation*}
\begin{equation*}
  \begin{aligned}
   x_{3}^{(k+1)}= \frac{3.5}{15.5} \cdot x_{2}^{(k+1)} + \frac{4}{15.5} \cdot x_{5}^{(k)}+ \frac{14}{9.5}\\
    \end{aligned}
\end{equation*}
\begin{equation*}
  \begin{aligned}
   x_{4}^{(k+1)}= \frac{2}{7} \cdot x_{1}^{(k+1)} + \frac{3}{7} \cdot x_{5}^{(k)}+ \frac{10}{7} \\
    \end{aligned}
\end{equation*}
\begin{equation*}
  \begin{aligned}
   x_{5}^{(k+1)}= \frac{5}{12} \cdot x_{2}^{(k+1)} + \frac{4}{12} \cdot x_{3}^{(k+1)} + \frac{3}{12} \cdot x_{4}^{(k+1)} - \frac{30}{12} \\
 \end{aligned}
\end{equation*}

Po\v cetna vrednost je $x^{0}=[0,0,0,0]^{T}$ i stajemo sa iteracijom kada se dve susedne iteracije poklope sa 4 decimala.

Za ovu metodu je bilo potrebno u programu pripremiti sistem jedna\v cina kako bi se primenile iteracije. Deo programa koji priprema vektore tipa "Jedna\v cina"\ :



\begin{lstlisting}[language=C++ ]
void SistemJednacina::noviSistemJednacina(){
	for (int i = 0; i < 5; i++) {
		for (int j = 0; j < 6; j++) {
			if (i != j) {
				if (j == 5) {
					jednacine_[i]->koeficijenti_[j] = jednacine_[i]->koeficijenti_[j] / jednacine_[i]->koeficijenti_[i];
				}
				else {
					jednacine_[i]->koeficijenti_[j] = (jednacine_[i]->koeficijenti_[j] / jednacine_[i]->koeficijenti_[i])*(-1);
				}
			}			
		}
		jednacine_[i]->koeficijenti_[i] = 0;
	}
}
\end{lstlisting}

Deo programa koji nalazi re\v senja i puni vektor "resenje\_Zajdel\_":

\begin{lstlisting}[language=C++ ]
void SistemJednacina::nalazenjeResenja2(){
	bool radi = true;
	double x1, x2, x3, x4, x5;
	
	while (radi) {
		for (double R : resenje_Zajdel_) {
			cout << R << "  ";
		}
		cout << endl;
		x1 = resenje_Zajdel_[1];
		x2 = resenje_Zajdel_[2];
		x3 = resenje_Zajdel_[3];
		x4 = resenje_Zajdel_[4];
		x5 = resenje_Zajdel_[5];

		for (int i = 1; i <= 5; i++) {
			resenje_Zajdel_[i] = jednacine_[i-1]->koeficijenti_[0] * resenje_Zajdel_[1] + jednacine_[i - 1]->koeficijenti_[1] * resenje_Zajdel_[2] + jednacine_[i - 1]->koeficijenti_[2] * resenje_Zajdel_[3] + jednacine_[i - 1]->koeficijenti_[3] * resenje_Zajdel_[4] + jednacine_[i - 1]->koeficijenti_[4] * resenje_Zajdel_[5] + jednacine_[i - 1]->koeficijenti_[5];
		}
		//if (x1 == resenje_Zajdel_[1] && x2 == resenje_Zajdel_[2] && x3 == resenje_Zajdel_[3] && x4 == resenje_Zajdel_[4] && x5 == resenje_Zajdel_[5]) {
		//	radi = false;
		//}

		if (uporedi(resenje_Zajdel_[1], x1) && uporedi(resenje_Zajdel_[2], x2) && uporedi(resenje_Zajdel_[3], x3) && uporedi(resenje_Zajdel_[4], x4) && uporedi(resenje_Zajdel_[5], x5)) {
			radi = false;
		}

		resenje_Zajdel_[0]++;
		//for (double R : resenje_Zajdel_) {
		//	cout << R << "  ";
		//}
		//cout << endl;
	
	}
}

bool SistemJednacina::uporedi(double trenutni_rezultat, double prethodni_rezultat, double razlika){
	if (fabs(trenutni_rezultat - prethodni_rezultat) < razlika) {
		return true;
	}
	return false;
}
\end{lstlisting}

Tabela Iteracije:

\begin{table}[h!]
\centering
 \begin{tabular}{||c c c c c c||} 
 \hline
 k & x_{1} & x_{2} & x_{3} & x_{4} & x_{5} \\ [0.5ex] 
 \hline\hline
 0 & 0 & 0 & 0 & 0 & 0\\ 
 1 & 1.26316 & -1.16746 & 0.639605 & 1.78947 & -2.32587\\ 
 2 & 0.33266 & -2.00537 & -0.149826 & 0.812529 & -3.18238\\ 
 3 & 0.906487 & -2.74273 & -0.537361 & 0.32369 & -3.741\\ 
 4 & 0.609531 & -3.18745 & -0.781941 & -0.000564235 & -4.08889\\ 
 5 & 0.424236 & -3.46552 & -0.934508 & -0.202601 & -4.30612\\ 
 6 & 0.308527 & -3.6391 & -1.02976 & -0.328757 & -4.44173\\ 
 7 & 0.236289 & -3.74746 & -1.08923 & -0.407517 & -4.5264\\ 
 8 & 0.19119 & -3.81512 & -1.12636 & -0.456688 & -4.57926\\ 
 9 & 0.163034 & -3.85736 & -1.14953 & -0.487368 & -4.61226\\ 
 10 & 0.145456 & -3.88373 & -1.16401 & -0.506552 & -4.63286\\ 
 11 & 0.134481 & -3.90019 & -1.17304 & -0.518517 & -4.64572\\ 
 12 & 0.127630 & -3.91047 & -1.17868 & -0.525987 & -4.65375\\ 
 13 & 0.123353 & -3.91689 & -1.1822 & -0.53065 & -4.65877\\ 
 14 & 0.120682 & -3.92089 & -1.1844 & -0.533562 & -4.6619\\ 
 15 & 0.119015 & -3.92339 & -1.18577 & -0.53538 & -4.66385\\ 
 16 & 0.117974 & -3.92496 & -1.18663 & -0.536515 & -4.66507\\
 17 & 0.117324 & -3.92593 & -1.18716 & -0.537223 & -4.66583\\
 18 & 0.116919 & -3.92654 & -1.1875 & -0.537665 & -4.66631\\
 19 & 0.116665 & -3.92692 & -1.18771 & -0.537941 & -4.6666\\
 20 & 0.116507 & -3.92716 & -1.18784 & -0.538114 & -4.66679\\
 21 & 0.116408 & -3.9273 & -1.18792 & -0.538222 & -4.6669\\
 22 & 0.116347 & -3.9274 & -1.18797 & -0.538289 & -4.66698\\
 23 & 0.116308 & -3.92745 & -1.188 & -0.538331 & -4.66702\\
 24 & 0.116284 & -3.92749 & -1.18802 & -0.538357 & -4.66705\\
 25 & 0.116269 & -3.92751 & -1.18803 & -0.538373 & -4.66707\\
 26 & 0.11626 & -3.92753 & -1.18804 & -0.538383 & -4.66708\\
 \hline
 \end{tabular}
\end{table}

\begin{equation*}
  \begin{aligned}
    x_{1}=0.1163
    \end{aligned}
\end{equation*}
\begin{equation*}
  \begin{aligned}    
    x_{2}=-3.9275
    \end{aligned}
\end{equation*}
\begin{equation*}
  \begin{aligned}
    x_{3}=-1.1880
    \end{aligned}
\end{equation*}
\begin{equation*}
  \begin{aligned}
    x_{4}=-0.5384
    \end{aligned}
\end{equation*}
\begin{equation*}
  \begin{aligned}
    x_{5}=-4.6671
    \end{aligned}
\end{equation*}
\end{aligned}
\end{equation*}

\section{Komentari}

Koeficijenti sistema ne ispunjavaju uslov stroge dijagonalne dominatnosti:
\begin{equation*}
  \begin{aligned}
   \mid a_{ii}\mid>\sum_{\substack{1<j<n, j\neq i}} \mid a_{ij}\mid,
  \end{aligned}
\end{equation*}
ta\v cnije u poslednjoj vrsti matrice sistema dijagonalni element jednak je zbiru preostalih elemenata te vrste ($\mid 12 \mid =\mid-5\mid + \mid-4\mid + \mid-3\mid $). Medjutim, uslov strogo dijagonalne dominantnosti nije potreban uslov da bi Gaus-Zajdelova metoda konvergirala. Uslov strogo dijagonalne dominantnosti obezbedjuje pre svega da pri preformulaciji jedna\v cina dijagonalni elementi kojima delimo budu razli\v citi od 0 \v sto u ovom slu\v caju ne moramo da se brinemo tj. nema potrebe za elementarnim transformacija sistema jedna\v cina. Broj iterativnih koraka \' ce biti manje ukoliko su elementi na dijagonali izrazito 
dominatniji tj. konvergencija \' ce biti br\v za(u ovom slu\v caju je spora konvergencija jer dijagonalni elementi nisu \v cak ni striktno dominantni).
\section{Dodatak}

JEDNA\v CINA.H
\begin{lstlisting}[language=C++ ]
#ifndef JEDNACINA_H
#define JEDNACINA_H

#include <iostream>
#include <string>
#include <vector>
using namespace std;

class Jednacina {
public:
	Jednacina(const string& ime = "jednacina");

	void citajJednacinu(string unos);
	void pisi() const;
	vector<double> koeficijenti_;

private:
	string ime_;
};

#endif
#pragma once


\end{lstlisting}
JEDNA\v CINA.CPP
\begin{lstlisting}[language=C++ ]
#include "Jednacina.h"

Jednacina::Jednacina(const string& ime) : ime_(ime) {}

void Jednacina::citajJednacinu(string unos){
	int poz = 0;
	int promenljiva = 1;
	string broj = "";
	float i;

	while (poz < unos.length()) {
		while (((unos[poz] >= '0' && unos[poz] <= '9') || unos[poz] == '.' || unos[poz] == '-') && poz < unos.length()) {
			broj += unos[poz];
			poz++;
		}
		if (poz >= unos.length()) { 
			while (promenljiva <= 5) {
				koeficijenti_.push_back(0);
				promenljiva++;
			}
			koeficijenti_.push_back(stod(broj));
			break; 
		}
		if (unos[poz+1] == promenljiva) {
			koeficijenti_.push_back(stod(broj));
			promenljiva++;
		}
		else {
			while ((int)(unos[poz + 1]-'0') > promenljiva) {
				koeficijenti_.push_back(0);
				promenljiva++;
			}
			koeficijenti_.push_back(stod(broj));
			promenljiva++;
		}
		poz += 3;
		broj = "";
		if(unos[poz-1] == '-'){ broj += '-'; }
	}
}

void Jednacina::pisi() const{
	for (float koeficijent : koeficijenti_) {
		cout << koeficijent << "	  ";
	}
	cout << endl;
	cout << endl;
	cout << endl;
}


\end{lstlisting}
SISTEMJEDNA\v CINA.H
\begin{lstlisting}[language=C++ ]
#ifndef SISTEMJEDNACINA_H
#define SISTEMJEDNACINA_H

#include "Jednacina.h"

#include <cmath>
#include <vector>
using namespace std;

class SistemJednacina {
public:
	SistemJednacina(const string& ime = "sistem jednacina");

	void citajJednacinu(string unos);
	void pisi() const;

	void pivotiranje(int);
	void noviMultiplikatori(int);
	void gausMetoda();
	void novaMatrica(int);
	void nalazenjeResenja();

	void gausZajdelovaMetoda();
	void noviSistemJednacina();
	void nalazenjeResenja2();
	bool uporedi(double trenutni_rezultat, double prethodni_rezultat, double razlika=0.00001f);

private:
	vector<Jednacina*> jednacine_;
	string ime_;
	vector<double> multiplikatori_;
	vector<double> resenje_Gaus_;
	vector<double> resenje_Zajdel_;
};

#endif

\end{lstlisting}
SISTEMJEDNA\v CINA.CPP
\begin{lstlisting}[language=C++ ]
#include "SistemJednacina.h"

SistemJednacina::SistemJednacina(const string& ime) : ime_(ime) {
	for (int i = 0; i < 6; i++) {
		resenje_Zajdel_.push_back(0);
	}
}

void SistemJednacina::citajJednacinu(string unos){
	Jednacina* jednacina = new Jednacina();

	jednacina->citajJednacinu(unos);
	jednacine_.push_back(jednacina);

}

void SistemJednacina::pisi() const{
	for (Jednacina* jednacina : jednacine_) {
		jednacina->pisi();
	}
}

void SistemJednacina::pivotiranje(int broj_kolone){
	int maks = broj_kolone;
	for (int i = maks; i+1< 5; i++) {
		if (abs(jednacine_[maks]->koeficijenti_[broj_kolone]) < abs(jednacine_[i + 1]->koeficijenti_[broj_kolone])) {
			maks = i + 1;
		}
	}
	if (maks != broj_kolone) {
		Jednacina* jednacina = new Jednacina();
		jednacina->koeficijenti_.push_back(jednacine_[maks]->koeficijenti_[0]);
		jednacina->koeficijenti_.push_back(jednacine_[maks]->koeficijenti_[1]);
		jednacina->koeficijenti_.push_back(jednacine_[maks]->koeficijenti_[2]);
		jednacina->koeficijenti_.push_back(jednacine_[maks]->koeficijenti_[3]);
		jednacina->koeficijenti_.push_back(jednacine_[maks]->koeficijenti_[4]);
		jednacina->koeficijenti_.push_back(jednacine_[maks]->koeficijenti_[5]);

		jednacine_[maks]->koeficijenti_[0] = jednacine_[broj_kolone]->koeficijenti_[0];
		jednacine_[maks]->koeficijenti_[1] = jednacine_[broj_kolone]->koeficijenti_[1];
		jednacine_[maks]->koeficijenti_[2] = jednacine_[broj_kolone]->koeficijenti_[2];
		jednacine_[maks]->koeficijenti_[3] = jednacine_[broj_kolone]->koeficijenti_[3];
		jednacine_[maks]->koeficijenti_[4] = jednacine_[broj_kolone]->koeficijenti_[4];
		jednacine_[maks]->koeficijenti_[5] = jednacine_[broj_kolone]->koeficijenti_[5];

		jednacine_[broj_kolone]->koeficijenti_[0] = jednacina->koeficijenti_[0];
		jednacine_[broj_kolone]->koeficijenti_[0] = jednacina->koeficijenti_[1];
		jednacine_[broj_kolone]->koeficijenti_[0] = jednacina->koeficijenti_[2];
		jednacine_[broj_kolone]->koeficijenti_[0] = jednacina->koeficijenti_[3];
		jednacine_[broj_kolone]->koeficijenti_[0] = jednacina->koeficijenti_[4];
		jednacine_[broj_kolone]->koeficijenti_[0] = jednacina->koeficijenti_[5];

	}
}

void SistemJednacina::noviMultiplikatori(int broj_vrste){
	int pocetni_red = broj_vrste - 1;
	for (int i = broj_vrste; i < 5; i++) {
		multiplikatori_.push_back(jednacine_[i]->koeficijenti_[broj_vrste-1]/ jednacine_[pocetni_red]->koeficijenti_[broj_vrste-1]);
	}
}

void SistemJednacina::gausMetoda(){
	int i = 5;
	/*pivotiranje(0);
	noviMultiplikatori(1);
	novaMatrica(1);
	multiplikatori_.clear();
	pisi();

	pivotiranje(1);
	noviMultiplikatori(2);
	novaMatrica(2);
	multiplikatori_.clear();
	pisi();

	pivotiranje(2);
	noviMultiplikatori(3);
	novaMatrica(3);
	multiplikatori_.clear();
	pisi();

	pivotiranje(3);
	noviMultiplikatori(4);
	novaMatrica(4);
	multiplikatori_.clear();
	pisi();*/
	for (int i = 0; i < 4; i++) {
		pivotiranje(i);
		noviMultiplikatori(i+1);
		novaMatrica(i + 1);
		cout << endl;
		cout << i+1 << ".korak";
		cout << "======================================================" << endl;
		cout << "multiplikatori: ";
		for (double mp : multiplikatori_) {
			cout << mp << "  ";
		}
		cout << endl;
		cout << "======================================================" << endl;
		multiplikatori_.clear();
		cout << "======================================================" << endl;
		pisi();
		cout << "======================================================" << endl;
	}
	nalazenjeResenja();

	cout << "Resenje sistema jednacina su: ";
	for (double R : resenje_Gaus_) {
		cout << "x" << i-- << ":" << R << "  ";
	}
	cout << endl;

}

void SistemJednacina::gausZajdelovaMetoda(){
	noviSistemJednacina();
	nalazenjeResenja2();
	
}

void SistemJednacina::noviSistemJednacina(){
	for (int i = 0; i < 5; i++) {
		for (int j = 0; j < 6; j++) {
			if (i != j) {
				if (j == 5) {
					jednacine_[i]->koeficijenti_[j] = jednacine_[i]->koeficijenti_[j] / jednacine_[i]->koeficijenti_[i];
				}
				else {
					jednacine_[i]->koeficijenti_[j] = (jednacine_[i]->koeficijenti_[j] / jednacine_[i]->koeficijenti_[i])*(-1);
				}
			}			
		}
		jednacine_[i]->koeficijenti_[i] = 0;
	}
}

void SistemJednacina::nalazenjeResenja2(){
	bool radi = true;
	double x1, x2, x3, x4, x5;
	
	while (radi) {
		for (double R : resenje_Zajdel_) {
			cout << R << "  ";
		}
		cout << endl;
		x1 = resenje_Zajdel_[1];
		x2 = resenje_Zajdel_[2];
		x3 = resenje_Zajdel_[3];
		x4 = resenje_Zajdel_[4];
		x5 = resenje_Zajdel_[5];

		for (int i = 1; i <= 5; i++) {
			resenje_Zajdel_[i] = jednacine_[i-1]->koeficijenti_[0] * resenje_Zajdel_[1] + jednacine_[i - 1]->koeficijenti_[1] * resenje_Zajdel_[2] + jednacine_[i - 1]->koeficijenti_[2] * resenje_Zajdel_[3] + jednacine_[i - 1]->koeficijenti_[3] * resenje_Zajdel_[4] + jednacine_[i - 1]->koeficijenti_[4] * resenje_Zajdel_[5] + jednacine_[i - 1]->koeficijenti_[5];
		}
		//if (x1 == resenje_Zajdel_[1] && x2 == resenje_Zajdel_[2] && x3 == resenje_Zajdel_[3] && x4 == resenje_Zajdel_[4] && x5 == resenje_Zajdel_[5]) {
		//	radi = false;
		//}

		if (uporedi(resenje_Zajdel_[1], x1) && uporedi(resenje_Zajdel_[2], x2) && uporedi(resenje_Zajdel_[3], x3) && uporedi(resenje_Zajdel_[4], x4) && uporedi(resenje_Zajdel_[5], x5)) {
			radi = false;
		}

		resenje_Zajdel_[0]++;
		//for (double R : resenje_Zajdel_) {
		//	cout << R << "  ";
		//}
		//cout << endl;
	
	}
}

bool SistemJednacina::uporedi(double trenutni_rezultat, double prethodni_rezultat, double razlika){
	if (fabs(trenutni_rezultat - prethodni_rezultat) < razlika) {
		return true;
	}
	return false;
}

void SistemJednacina::novaMatrica(int broj_vrste){
	int pocetni_red = broj_vrste-1, k=0;
	for (int i = broj_vrste; i < 5; i++) {
		for (int j = broj_vrste-1; j < 6; j++) {
			jednacine_[i]->koeficijenti_[j] = jednacine_[i]->koeficijenti_[j] - multiplikatori_[k] * jednacine_[pocetni_red]->koeficijenti_[j];
		}
		k++;
	}
}

void SistemJednacina::nalazenjeResenja(){
	double sum = 0;
	
	for (int i = 4; i >= 0; i--) {
		sum = 0;
		if (i == 4) {
			resenje_Gaus_.push_back(jednacine_[i]->koeficijenti_[i+1] / jednacine_[i]->koeficijenti_[i]);
		}
		else {
			for (int j = i+1, k=1; j<5 ; j++,k++ ) {
				sum += (jednacine_[i]->koeficijenti_[j]* resenje_Gaus_[(int)(resenje_Gaus_.size())-k]);
			}
			resenje_Gaus_.push_back(1/(jednacine_[i]->koeficijenti_[i])*(jednacine_[i]->koeficijenti_[5] -sum));
		}
	}
}

\end{lstlisting}
MAIN.CPP
\begin{lstlisting}[language=C++ ]
#include "SistemJednacina.h"

#include "string"
using namespace std;

int main() {
	string nova_jednacina;
	int izbor;
	bool radi = true;

	SistemJednacina* sistem_jednacina = new SistemJednacina();

	while (radi) {
		cout << "Odaberite opciju:\n1. Unos nove jednacine\n2. Kraj unosa " << endl;
		cin >> izbor;

		try {
			switch (izbor) {
			case 1: {  //Ucitavanje nove jednacine
				string putanja;

				//cout << izbor << "\nNova jednacina => " << endl;    //////////////////////////
				//cin >> nova_jednacina;								/////////////////////////

				//sistem_jednacina->citajJednacinu(nova_jednacina);////////////////////////////////////
				sistem_jednacina->citajJednacinu("9.5i1-2.5i2-2i4=12");
				sistem_jednacina->citajJednacinu("-2.5i1+11i2-3.5i3-5i5=-16");
				sistem_jednacina->citajJednacinu("-3.5i2+15.5i3-4i5=14");
				sistem_jednacina->citajJednacinu("-2i1+7i4-3i5=10");
				sistem_jednacina->citajJednacinu("-5i2-4i3-3i4+12i5=-30");
				sistem_jednacina->pisi();
				cout << "\nZavrsen unos " << endl;
				
				break;
			}
			case 2: {  //Kraj unosa
				radi = false;
				break;
			}
			default:
				cout << "Netacan unos\n\n";
			}
		}
		catch (string izuzetak) {
			//prihvatiGresku(izuzetak);
		}
	}

	if (izbor) { radi = true; }

	while (radi) {
		cout << "=======================================================" << endl;
		cout << "Molimo Vas, odaberite nacin resavanja zadatka:\n1. Gausova metoda eliminacije sa izborom pivota\n2. Gaus-Zajdenlova metoda\n3. Prikaz prosirene matrice sistema jednacine\n4. Kraj programa\n " << endl;
		cout << "=======================================================" << endl << endl;
		cin >> izbor;  //izbor opcija

		switch (izbor) {

		case 1: {		//Gausova metoda eliminacije sa izborom pivota
			sistem_jednacina->gausMetoda();
			//sistem_jednacina->pisi();
			break;
		}
		case 2: {		//Gaus-Zajdenlova metoda
			sistem_jednacina->gausZajdelovaMetoda();
			break;
		}	
		case 3: {		//Prikaz proserene matrice koeficijenata
			sistem_jednacina->pisi();
			break;
		}
		case 4: {		//Kraj programa
			cout << "Kraj programa" << endl;
			radi = false;
			break;
		}
		default:
			cout << "Netacan unos\n\n";
		}
	}

	delete sistem_jednacina;

	return 0;
}

\end{lstlisting}


\end{document}